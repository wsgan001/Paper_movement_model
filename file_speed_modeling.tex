\subsection{Speed Distribution}
\label{section_speed_modeling}
\begin{figure}[!h]
\centering
\begin{tabular}
[c]{cc}
\multicolumn{2}{c}{6:00-8:00}\\
\epsfysize=1.2in\epsfbox{figures/evalue/fitspeed6_0.eps} &
\epsfysize=1.2in\epsfbox{figures/evalue/fitspeed6_1.eps} \\
\multicolumn{2}{c}{11:00-13:00}\\
\epsfysize=1.2in\epsfbox{figures/evalue/fitspeed11_0.eps} &
\epsfysize=1.2in\epsfbox{figures/evalue/fitspeed11_1.eps} \\
\multicolumn{2}{c}{17:00-19:00}\\
\epsfysize=1.2in\epsfbox{figures/evalue/fitspeed17_0.eps} &
\epsfysize=1.2in\epsfbox{figures/evalue/fitspeed17_1.eps} \\
\multicolumn{2}{c}{22:00-24:00}\\
\epsfysize=1.2in\epsfbox{figures/evalue/fitspeed22_0.eps} &
\epsfysize=1.2in\epsfbox{figures/evalue/fitspeed22_1.eps} \\
(a) vacant status & (b) occupied status \\
\end{tabular}
\caption{fit result for taxi speed distribution}\label{figure_fitspeed_varied_with_time}
\end{figure}

To obtain the speed distribution of each status, we fit the cumulative instantaneous speed distribution to get the cumulative probability distribution function, and then take a derivative with it to obtain the speed probability distribution. From Fig.~\ref{figure_fitspeed_varied_with_time}, the instantaneous speed distribution shows exponential law expcet for that of occupied status from 22:00 to 24:00. Given that, we fit the speed distribution by an exponential function $f_1(x)$, and fit the cumulative speed distribution of occupied status from 22:00 to 24:00 by a linear function $f_2(x)$, as formular \ref{formular_ccdf_speed}. In order to eliminate the influence caused by the weekend, we remove the speed distribution data such as the data of occupied status from 6:00 to 8:00, to generalize the fitting results. 
\begin{equation}\label{formular_ccdf_speed}
\left\{
\begin{array}{ll}
f_1(x) = 1-1/exp(-ax^b-c)\\
f_2(x) = ax+b
\end{array}
\right.
\end{equation}
Here, $f_i(x)$ is the function form for the instantaneous speed distribution. The \emph{root mean square} (rms) of residuals for each fit are reported in Table~\ref{table_rms}. The smaller rms of residuals means better fitting. In this table, the values are all less than $0.025$, showing good similarity.
\begin{table}[!h]
\caption{Parameters and rms of residuals of fitting curves}\label{table_rms}
\centering
\begin{tabular}{c|c|c}
  \hline
  Time period  & vacant status & occupied status \\
  \hline
6:00-8:00   &0.0129207 & 0.019818 \\
11:00-13:00 &0.00866176 & 0.0204889 \\
17:00-19:00 &0.0176578 & 0.0105868 \\
22:00-24:00 &0.0154822 & 0.0240426 \\
  \hline
\end{tabular}
\end{table}

