\subsection{Trace Dataset: Beijing Taxi Traces}
\label{section_trace_data}

A real-world GPS data set is used in this paper, which was generated by $12,455$ taxis in Beijing, China within 5 days from March 3rd,2011 to March 7th,2011.
Each row includes a base station ID, company name, taxi ID ($id$), timestamp ($t$), current location ($l$, including longitude and latitude), speed, event, status, et al. Of all the fields, the taxi ID, time stamp, and current location, status and event are used in this paper. Note that GPS traces from taxis have been used recently for inferring human mobility \cite{Ganti} and modeling city-scale traffics \cite{Aslam}. Therefore, we believe that they are suitable to characterize the contact patterns among vehicles in large-scale urban scenario.

\begin{table}[!h]
\caption{Explanation of Events and Status}\label{table_event_detail}
\centering
\begin{tabular}{l|l}
  \hline
  Event & Explanation \\
  \hline
  0(drop) & A taxi's status change to vacant.\\
  \hline
  1(load) & A taxi's status change to occupied.\\
  \hline
  2 & Set up defense.\\
  \hline
  3 & Cancel defense.\\
  \hline
  4 & No event happened.\\
  \hline
  \hline
  Status & Explanation \\
  \hline
0(vacant) & A taxi is vacant. \\
    \hline
1(occupied) & A taxi is occupied. \\
    \hline
2 & A taxi is setting up defense. \\
    \hline
3 & Stop running.\\
  \hline
\end{tabular}
\end{table}


Especially, there are five types of events and four types of statuses. The explanations are as table \ref{table_event_detail}. We only discuss the vacant status and occupied status in this paper. Accordingly we utilize the load-event and drop-event as well.






