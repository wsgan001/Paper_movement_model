\section{Related Works}
\label{section_related_works}

For the free space scenario, the random way point (RWP)\cite{broch1998performance} movement model is the most commonly used. The movement model identified a pause time, speed range from zero to the maximum, and movement area where the model select a random destination. Amit Kumar Saha, el at. \cite{SahaJohnson-91} found that RWP in many cases the Random Waypoint mobility model is a good approximation of the vehicular mobility model based on real street maps.
The constrained mobility models show closer relevance to the realistic. Some models \cite{HuangZhu-88,SahaJohnson-91,MartinezCano-87,ChoffnesBustamante-93} take the geographic structure into consideration. 

In this section,  we sum up the researches on mobility models.Mobility model can be classified into free space and constrained models\cite{LuChen-104,AhmedKarmakar-106} based on the degree of randomness.
Random walk, Random Waypoint and Random Direction mobility model are three classical free style mobility models. Those models defined simple mobility patterns, which is good for us to create mobility models and analysis. But they also have notable disadvantages, e.g., they  are out of reality, because too many factors are ignored. 

In order to increase the degree of reality, mobility models are constrained or relayed in many respects. Some researchers build models based on geographic models. Manhattan models \cite{FBaimanhattan} are a typical models which models the city as a Manhattan style grid, with a uniform block size across the simulation area, while all streets are two-way with a lane in each direction which constrained car movements \cite{MartinezCano-87}, and nodes can move straight forward or turn direction at a cross road.
Other models import the map information into mobility models. In 2004, Saha, A.K \cite{SahaJohnson-91} model the vehicle networks based on the real map. It compared with the RWP models, a frequently used model in vehicular networks, to find out the difference between the RWP and real traces. In 2005, literature \cite{ChoffnesBustamante-93} proposed a comprehensive models on wireless vehicular networks and transportation. This paper simplifies the real map to estimate the network performance in ad hoc and proposes the mobility model STRAW. Besides, some other mobility focus on the microscopic characteristics of mobility, they introduce the transportation features into mobility, such as the traffic lights and accretion on some road segments. The information of the geography can improve the reality. In 2007, Atulya Mahajan, et al. \cite{MahajanPotnis-102} accounted for the street layout, traffic rules, multi-lane roads, acceleration-deceleration, and radio frequent (RF) attenuation due to obstacles, and further evaluated the synthetic maps by comparing with real maps. In 2008, David R. Choffnes, et al.\cite{ChoffnesBustamante-93} developed their movement model based on a realistic vehicular traffic model on road defined by real street map data. but too many microscopic detail can not be applied to scenarios with large scale nodes.

In recent years, vehicular sensors or handheld devices spread rapidly, that makes it possible to collect and analyze the real traces of large amount of nodes and helps us to improve the traffic and network macroscopically.
In 2010 and 2012, Huang H, et al. \cite{HuangZhu-88} proposed mobility models based on taxi trace data in Shanghai, China. They designed three parameters, i.e., turn probability, road section speed and travel pattern, which can be estimated by analyzing the data statistically. But it is complicated to re-implement this model, for the model is strongly related to the map they simplified from the real map.
To the best of our knowledge, our work is original to develop mobility models by investigating taxi behavior, time and geographic feathers of different statuses.
