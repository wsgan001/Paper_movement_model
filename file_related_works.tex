\section{Related Works}
\label{section_related_works}

In recent years, vehicular ad hoc networks (VANETs) have drawn growing interest. Mobility model is crucial to simplify scenario and capture the main characteristics of VANETs by simulation, even though real data sets are available. We briefly review the mobility models below.
Based on the restrictions of environmental attributes, mobility model can be classified into free space and constrained models\cite{LuChen-104,AhmedKarmakar-106}, while the restrict models can also be further categorized into models restricted by geographic structure, trace data, or both.
For the free space scenario, the random way point (RWP)\cite{broch1998performance} movement model is the most commonly used. The movement model identified a pause time, speed range from zero to the maximum, and movement area where the model select a random destination. Upon reaching the destination, the node pauses again for a pause time, selects another destination, and repeats the previous procedure again \cite{broch1998performance} . Amit Kumar Saha, el at. \cite{SahaJohnson-91} found that RWP in many cases the Random Waypoint mobility model is a good approximation of the vehicular mobility model based on real street maps.
The constrained mobility models have a close relevance to the real-world movement. Literature \cite{MayerWaldhorst-108} demonstrated that graph structure is close related with inter-contact time distribution in both random and social mobility on grid-based graphs. Some models \cite{SahaJohnson-91,PengDong-101,HuangZhu-88,MartinezCano-87,ChoffnesBustamante-93} take the geographic structure into consideration. Manhattan model (MM) models the city as a Manhattan style grid, with a uniform block size across the simulation area, while all streets are two-way with a lane in each direction which constrained car movements \cite{MartinezCano-87}. Downtown models add traffic density to the manhattan models to increase the accuracy in simulation. In 2007, Atulya Mahajan, et al. \cite{MahajanPotnis-102} accounted for the street layout, traffic rules, multi-lane roads, acceleration-deceleration, and radio frequent (RF) attenuation due to obstacles, and further evaluated the synthetic maps by comparing with real maps. In 2008, David R. Choffnes, et al.\cite{ChoffnesBustamante-93} developed their movement model based on a realistic vehicular traffic model on road defined by real street map data.

With the development in Vehicular communication equipment, large amount of data can be collected and utilized by researchers. Many mobility models take advantage of the trace data to achieve better accuracy and performance. In 2007, Zhang X, Kurose J, Levine B N, et al. \cite{ZhangKurose-98} analyzed the bus trace taken from UMass DieselNet, consisting of Wi-Fi nodes attached to buses. They found that inter-contact times aggregated at a route level exhibit periodic behavior. Thus, they construct generative route-level models that capture the above behavior. However, the mobility model based on bus data with periodic behavior may lack in its universality in VANET.

Other researchers considered the graph factor and trace comprehensively. In 2006, by extracting wireless network traces required from 10,000 users at Dartmouth College over several years, Kim, al et. \cite{KimKotz-99} developed a movement model based on their discoveries that the speed and pause time follow a log-normal distribution and the direction of movements closely reflects the direction of roads and walkways \cite{KimKotz-99}. While the user data can reflect less mobility, we cannot apply it to VANET directly. In 2010 and 2012, Huang H, et al. \cite{HuangZhu-88,HuangZhang-105} proposed mobility models based on taxi trace data in shanghai, China. They designed three parameters, i.e., turn probability, road section speed and travel pattern, which can be estimated by analyzing the data statistically.
Whereas the mobility models existing have captured many characteristics in VANETs, the phenomenon that the attributes of vehicles, i.e., speed, turn probability, varies a lot from area to area, is ignored, limiting the accuracy of models. To the best of our knowledge, our work is original to develop mobility models by investigating the inequality in geographical distribution.


