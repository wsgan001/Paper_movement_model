\section{Introduction}
\label{section_introduction}

In vehicle ad hoc networks (VANETs) \cite{4068700}, realistic movement model is an important way to improve route planning, control traffic situations, or solve the vehicle-to-vehicle communication problems. However, movement models will influence simulation performance, since movement model defines the nodal movement pattern including speed and direction. Whereas. large amounts of data are difficult to utilized directly. It is necessary to work on realistic mobility models. Some researchers \cite{KimKotz-99,HuangZhu-88} modeled the vehicular movement, extracting different feathers from real data sets. But taxi status is ignored in the previous works.
In this work, a STatus and Region Aware Taxi mobility model, START, is proposed based on the real taxi GPS data. Two assumptions are introduced in section \ref{section_assumptions_anlysis}. We assume that the taxi behavior and geographic feathers are related with different status. They are validated to be reasonable by the statistical analysis of the data set.
START is modeled based on these assumptions. In the macro scope, instead of simply dividing the area into coarse-grain regions, we divide the area into two set of regions according to the passenger load or drop event density. When a taxi take a passenger, the current region will be selected in the region set of load-event and the destination region, where the drop-event happens, will be selected in the region set of drop-event. We investigate the relationship between load-event regions and drop-event regions. Pathes from the sources to destinations will be found by Dijkstra algorithm. For microscope, the \emph{speed} for the two taxi statuses are discussed respectively.
Simulations are carried out to compare the similarity of node trace characteristics and contact characteristics. The results show that our mobility model has a good approximation with the real scenario.
The rest of our paper is organized as follows: %%Section \ref{section_related_works} provides an overview of related works on mobility models.
 Section \ref{section_assumptions_anlysis} proposes two assumptions which are further validated by statistical results of real data. Section \ref{section_modeling} presents the modeling process. Simulation results are demonstrated in Section \ref{section_model_varification}. Finally, Section \ref{section_conclusion} concludes this paper.


