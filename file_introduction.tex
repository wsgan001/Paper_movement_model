\section{Introduction}
\label{section_introduction}

In vehicle ad hoc networks (VANETs) \cite{4068700}, realistic movement model is an important way to improve route planning, control traffic situations, or solve the vehicle-to-vehicle communication problems. However, movement models will influence simulation performance, since movement model defines the nodal movement pattern including speed and direction. It is necessary to work on realistic mobility models.There are some obstacles to create realistic mobility. Firstly, it is difficult to utilize large amount of data directly, because simulation scenarios are changeable. Some researchers \cite{KimKotz-99,HuangZhu-88} modeled the vehicular movement, extracting different feathers from real data sets. But taxi status is ignored in the previous works.

In this work, a STatus and Region Aware Taxi mobility model, START, is proposed based on the real taxi GPS data, which involves 12,455 taxis in Beijing, China and 74,175,360 records from March 3rd, 2011 to March 7th,2011. Four taxi statuses from 0 to 3 are given by the data set, taxi status 0 (\emph{vacant status}) and status 1 (\emph{occupied status}) are considered. The other two statuses(defense status and stop running status) will not be discussed in this paper, because the amounts of data is small and the behavior characteristics are not certain.
Two assumptions, one assumes that the taxi behavior differ with the taxi status and the other assumes that taxi movement has geographic features, are proposed in section \ref{section_assumptions_anlysis}. They are validated to be reasonable by the statistical analysis of the data set. The mobility model is developed on microscopic and macroscopic aspects. For microscope, the \emph{speed} and \emph{ duration } for the two taxi statuses are discussed respectively. For an instance, \emph{speed} distribution for vacant status indicates the probability for a nodes running at a certain speed, so that the speed of nodes can be assigned according to this rule.
In the macro scope, instead of simply dividing the area into squares, we cluster the area according to the node density. By dividing entire area into grids, cells adjacent to each other and with higher node density are grouped into dense regions while other girds are classified into one sparse region. Then the transition probability between regions are calculated, so that the macroscopic movement can be defined.
Simulations are carried out to compare the similarity of node trace characteristics and contact characteristics. In order to estimate the assumption that the taxi statuses cannot be ignored, a simplified model S-START based on START is implemented, which ignore the taxi status difference. The performance of START, S-START, Shortest Path movement model and Random Way Point model are compared with that of real trace data.
The results show that the taxi status have obvious effect on the taxi behavior and further influence the simulation results. Our mobility model has a good approximation with the real scenario.

The rest of our paper is organized as follows: Section \ref{section_related_works} provides an overview of related works on mobility models. Section \ref{section_assumptions_anlysis} proposes two assumptions which are further validated by statistical results of real data. Section \ref{section_modeling} presents the modeling process. Simulation results are demonstrated in Section \ref{section_model_varification}. Finally, Section \ref{section_conclusion} concludes this paper.

