
\subsection{Overview of START}

Movement model defines the mobility pattern of nodes, which can be represented as a collection of path segments, say $Paths:<p_1,p_2,...,p_n>$.  To generate a $p_i$, START takes two steps: destination selection and moving process (from current location to the selected destination).

\textbf{Destination Selection:} In START, the selection of a destination of a node is closely related to not only its current location but also its current status. We divide the area into regions by the density of passenger load/drop events, respectively. We will show how to divide the area in the next subsection. For a time period t, let $\textbf{R}^{load}=\{R_i^{load}\}$ and $\textbf{R}^{drop}=\{R_j^{drop}\}$ denote the set of regions of load and drop events, respectively. We assume that $\bigcup\{R_i^{load}\}=\bigcup\{R_j^{drop}\}$ which is the whole area.
Then, two transition probability matrixes are calculated: one is the transition probability from a passenger drop region $R_i^{drop}$ to a passenger load region $R_j^{load}$, while the other is the transition probability from a passenger load region $R_i^{load}$ to a passenger drop region $R_j^{drop}$.
If the status of a taxi changes to being vacant, its current location locates in a $R_i^{drop}$. Consequently, a destination region in $\textbf{R}^{load}$ will be selected by querying the transition matrix from $\textbf{R}^{drop}$ to $\textbf{R}^{load}$. Then, START will randomly select a map node in the region as the destination. As to the status of a taxi changing to being occupied, the destination selection process is similar except for that the transition matrix from $\textbf{R}^{load}$ to $\textbf{R}^{drop}$ is used instead. In summary, during this process, the destination of drop/load location is randomly selected based on the region transition matrix corresponding to the current status.

\textbf{Moving Process:} When the source location (current location) and destination location are given or selected, the next step is to find a path to connect them. To simplify the process, we adopt the Dijkstra algorithm, which will find a shortest path from the source to the destination based on the map. The speed of the path then is assigned to $speed$ based on the current status. Here, the value of $speed$ is drawn from the average speed distribution of corresponding status, which will be introduced in the last subsection of this section.


