
Movement model defines the mobility pattern of nodes, which can be represented as a collection of pathes��$Paths:<p_1,p_2,...,p_n>$, so dose START. A $p_i$ includes two steps, destination selection and moving process from source location to destination.

\textbf{Destination selection:} In START, to select a destination of a node is closely related to note only its current location but also its current status. Dividing the area into regions by the density of passenger on/off events or loading passenger events respectively, two transition probability matrixes are calculated, one is the probability from passenger off event regions $\{REGION_{m,off}\}$ to passenger on events regions $\{REGION_{n,on}\}$. Note that $\bigcup\{REGION_{m,off}\}=\bigcup\{REGION_{n,on}\}=AREA$. If the status of a taxi changes to vacant, its current location determine a $REGION_{i,off}$. Consequently, a destination region in $\{REGION_{n,on}\}$ will be selected by querying the transition matrix from $\{Region_{m,off}\}$ to $\{Region_{n,on}\}$. Then, START will randomly select a map node in the region as the destination. As to the status of a texi changing to Occupied, the destination selection process is similar. During this process, the region transition matrix will be utilized according to the current status.

\textbf{Moving process:} When the source location (current location) and destination location is given, next step is to find a path and set the speed. To simplify, we adopt the Dijkstra algorithm ,which will find a shortest path from source to the destination, to route on map. The speed is assigned by the average speed distribution of the corresponding status.

Based on the design above, we model the movement on the speed, duration and region transition matrix respectively.



