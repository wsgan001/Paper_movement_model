
Movement model defines the mobility pattern of nodes, which can be represented as a collection of pathes, say $Paths:<p_1,p_2,...,p_n>$, so dose the START. A $p_i$ takes two steps to adopt, destination selection and moving process from source location to destination.

\textbf{Destination selection:} In START, to select a destination of a node is closely related to note only its current location but also its current status. Dividing the area into regions by the density of passenger load/drop events or loading passenger events respectively, two transition probability matrixes are calculated, one is the probability from passenger drop event regions $\{REGION_{m,drop}\}$ to passenger on events regions $\{REGION_{n,load}\}$. Note that $\bigcup\{REGION_{m,drop}\}=\bigcup\{REGION_{n,load}\}=AREA$. If the status of a taxi changes to vacant, its current location determine a $REGION_{i,drop}$. Consequently, a destination region in $\{REGION_{n,load}\}$ will be selected by querying the transition matrix from $\{Region_{m,drop}\}$ to $\{Region_{n,load}\}$. Then, START will randomly select a map node in the region as the destination. As to the status of a texi changing to Occupied, the destination selection process is similar. During this process, the region transition matrix will be utilized according to the current status.

\textbf{Moving process:} When the source location (current location) and destination location is given, next step is to find a path. To simplify, we adopt the Dijkstra algorithm ,which will find a shortest path from source to the destination, to route on map. The speed of the path should be assigned as the $\overline{speed}$, which is adopted by the current $\overline{speed}$  distribution of corresponding status introduced in the following section.

Based on the design above, we model the movement on the speed, duration and region transition matrix respectively.



