
\subsection{Overview of START}

Movement model defines the mobility pattern of nodes, which can be represented as a collection of path segments, say $Paths:<p_1,p_2,...,p_n>$.  To generate a $p_i$, START takes two steps: destination selection and moving process (from current location to the selected destination).

\textbf{Destination Selection:} In START, the selection of a destination of a node is closely related to not only its current location but also its current status.
A travel path of a taxi can be simplified as a multi-hop process, in which a hop indicates an load/drop event happened.
  Seeing that, we define a \emph{ region transition probability} to figure out the probability of the next hop falling in a certain region from the current region. Particularly, when two successive events are different: one load and one drop.
 We divide the area into regions by the density of passenger load/drop events at different time, respectively. We will show how to divide the area in the next subsection. For a time period t, let

\begin{equation}
  \textbf{R}^{load}_t = \{R_{i,t}^{load}\}
\end{equation}
\begin{equation}
   \textbf{R}^{drop}_t = \{R_{j,t}^{drop}\}
\end{equation}

denote the set of regions of load and drop events, respectively. $i$ and $j$ are the region id of the load and drop event regions and t is a integer which denotes the hour of current time, such as t=7 means the time is in the range of 7:00:00-7:59:59. The union of the load or drop region set for a t is the whole area, that is $\bigcup\{R_{i,t}^{load}\}=\bigcup\{R_{j,t}^{drop}\} = Area $.
Then, for every time period, two transition probability matrixes are calculated: one is the transition probability from a passenger drop region $R_{i,t}^{drop}$ to a passenger load region $R_{j,t}^{load}$ or $R_{j',t+1}^{load}$, while the other is the transition probability from a passenger load region $R_{j,t}^{load}$ to a passenger drop region $R_{i,t}^{drop}$ or $R_{i',t+1}^{drop}$. From the status duration distribution, we find that the more than 95\% status duration for a taxi status is no longer than one hour. In this case, the transition probability from a region in time t to another region at time t’, where t’ is larger than t+1,  will be ignored. 
If the status of a taxi changes to being vacant, its current location locates and time in a $R_{i,t}^{drop}$. Consequently, a destination region in $\textbf{R}_{t}^{load}$ or $\textbf{R}_{t+1}^{load}$ will be selected by querying the transition matrix. Then, START will randomly select a map node in the region as the destination. As to the status of a taxi changing to being occupied, the destination selection process is similar except for that the transition matrix from $\textbf{R}_{t}^{load}$ to $\textbf{R}_t^{drop}$ or $\textbf{R}_{t+1}^{drop}$ is used instead. In summary, during this process, the destination of drop/load location is randomly selected based on the region transition matrix corresponding to the current status.

\textbf{Moving Process:} When the source location (current location) and destination location are given or selected, the next step is to find a path to connect them. To simplify the process, we adopt the Dijkstra algorithm, which will find a shortest path from the source to the destination based on the map. The speed of the path then is assigned to $speed$ based on the current status. Here, the value of $speed$ is drawn from the average speed distribution of corresponding status, which will be introduced in the last subsection of this section.


